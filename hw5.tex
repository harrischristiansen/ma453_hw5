\documentclass[11pt]{article}
\input{headers5}

\usepackage{fancyhdr}   
\pagestyle{fancy}      
\lhead{MA453 Spring 2018 - Homework 5}               
\rhead{Harris Christiansen (christih@purdue.edu)}

\usepackage{mathrsfs}
\usepackage[strict]{changepage}  
\newcommand{\nextoddpage}{\checkoddpage\ifoddpage{\ \newpage\ \newpage}\else{\ \newpage}\fi}

\begin{document}

\title{Homework 5}
\date{p86 B.1(b), p87 C.1(a,e), p97 A.3, p101 H.5 and I.3}
\maketitle

\thispagestyle{fancy}  
\pagestyle{fancy}      

\begin{enumerate}

%%% Problem p86 B.1(b)
\item {\bfseries p86 B.1(b)} Compute $\alpha^{-1}, \alpha^2, \alpha^3, \alpha^4, \alpha^5$ where $\alpha = (1234)$
  
	{\bfseries Solution.}
	
	\begin{tabular}{ l l }
	
	$\alpha^{-1} = \begin{pmatrix}
		1 & 2 & 3 & 4 \\
		4 & 1 & 2 & 3
	\end{pmatrix}
	= (1432)$ &
	$\alpha^2 = \begin{pmatrix}
		1 & 2 & 3 & 4 \\
		3 & 4 & 1 & 2
	\end{pmatrix}
	= (13)(24)$ \\
	
	$\alpha^3 = \begin{pmatrix}
		1 & 2 & 3 & 4 \\
		4 & 1 & 2 & 3
	\end{pmatrix}
	= (1432)$ &
	$\alpha^4 = \begin{pmatrix}
		1 & 2 & 3 & 4 \\
		1 & 2 & 3 & 4
	\end{pmatrix}
	= (1)(2)(3)(4)$
	
	\end{tabular}
	
	$\alpha^5 = \begin{pmatrix}
		1 & 2 & 3 & 4 \\
		2 & 3 & 4 & 1
	\end{pmatrix}
	= (1234)$

%%% Problem p87 C.1(a)
\item {\bfseries p87 C.1(a)} Determine if the permutation is even or odd. Justify your answer.
	
	$\pi = \begin{pmatrix}
		1 & 2 & 3 & 4 & 5 & 6 & 7 & 8 \\
		7 & 4 & 1 & 5 & 6 & 2 & 3 & 8
	\end{pmatrix}$
  
	{\bfseries Solution.}
	
	$\pi = \begin{pmatrix}
		1 & 2 & 3 & 4 & 5 & 6 & 7 & 8 \\
		7 & 4 & 1 & 5 & 6 & 2 & 3 & 8
	\end{pmatrix}
	= (173)(2456) = (17)(37)(24)(45)(56)$
	
	Since this requires 5 transpositions, the permutation $\pi$ is odd.

%%% Problem p87 C.1(e)
\item {\bfseries p87 C.1(e)} Determine if the permutation is even or odd. Justify your answer.

	$(123)(2345)(1357)$
  
	{\bfseries Solution.}
	
	$(123)(2345)(1357) = \begin{pmatrix}
		1 & 2 & 3 & 4 & 5 & 6 & 7 \\
		5 & 4 & 3 & 7 & 2 & 6 & 1
	\end{pmatrix} = (17)(14)(12)(15)$
	
	Since this requires 4 transpositions, the permutation $(123)(2345)(1357)$ is even.
  
\newpage

%%% Problem p97 A.3
\item {\bfseries p97 A.3} Let $G_1, G_2,$ and $G_3$ be groups, and let $f : G_1 \rightarrow G_2$ and $g : G_2 \rightarrow G_3$ be isomorphisms. Prove that $g \circ f : G_1 \rightarrow G_3$ is an isomorphism.
  
	{\bfseries Solution.}
	\begin{proof}
          Let $x,y \in G_1$ be two elements.
          By the property of an isomorphism we know that $f(xy) = f(x)f(y)$, where $xy$ represents the group action and $f(x), f(y) \in G_2$.
          
          By the properties of $g$ being an isomorphism we know that it's action on any two elements in $G_2$ preserve this same property. We can show:
          \begin{align*}
            (g \circ f)(xy) &= g(f(xy)) \\
            &= g(f(x)(y)) \\
            &= g(f(x))g(f(y))
          \end{align*}
          where $g(f(x)), g(f(y)) \in G_3$, therefor the first quality of an isomorphism is preserved.
          
          In previous homeworks we have shown that the composition of bijective functions (one to one and onto) are necessarily preserved by composition.
          Therefore the composition of isomorphisms must preserve this quality as well.
          That is, for each isomorphism $f,g \exists f^{-1}, g^{-1}$ 
          such that for any element $\alpha, \in G_3$ we can say: 
          \begin{align*}
            (f^-1 \circ g^-1)(g \circ f)(\alpha) &= f^-1(g^-1(g(f(\alpha)))) \\
            &= f^-1(f(\alpha)) \\
            &= \alpha 
          \end{align*}
          
          Therefore both properties of an isomorphism are preserved and the composition is an isomorphism.
	\end{proof}

%%% Problem p101 H.5
\item {\bfseries p101 H.5} Let $c$ be a fixed element of $G$. Let $H$ be a group with the same set as $G$, and with the operation $x * y = xcy$. Prove that the function $f(x) = c^{-1}x$ is an isomorphism from $G$ to $H$.
  
	{\bfseries Solution.}
	\begin{proof}
		First let us show that it preserves the multiplicative property. Let $x, y \in G$ be two elements. 
		\begin{align*}
			f(xy) &= c^-1xy \\
			&= c^-1xcc^{-1}y \text{ as $cc^{-1}$ is the identity element by definition of a group} \\
			&= f(x)f(y) 
		\end{align*}
		Therefore the first quality of an isomorphism is preserved.
      
		For bijection, let us define the inverse function $f^{-1}$ to be $f^{-1}(x) = cx$.
		This can be used to show that bijection is preserved:
		\begin{align*}
			(f^{-1} \circ f)(x) &= f^{-1}(c^{-1}x) \\
			&= cc^{-1}x \\
			&= x
		\end{align*}
		Therefore the bijection is preserved for all elements in $G,H$ and the function is an isomorphism.
	\end{proof}

%%% Problem p101 I.3
\item {\bfseries p101 I.3} If $G$ is any group, and $a$ is any element of $G$, prove that $f(x) = axa^{-1}$ is an automorphism of $G$.
  
	{\bfseries Solution.}
	\begin{proof}
		Property 1: (Let $x,y \in G$)
		\begin{align*}
			f(xy) &= axya^{-1} \\
			&= axaa^{-1}ya^{-1} \\
			&= f(x)f(y)
		\end{align*}
		So the first property is preserved.
		
		Property 2: (Let $x \in G$ and $f^{-1}(x) = a^{-1}xa$)
		\begin{align*}
			(f^{-1} \circ f)(x) &= f^{-1}(axa^{-1}) \\
			&= a^{-1}axa^{-1}a \\
			&= exe \\
			&= x
          \end{align*}
		Therefore both properties are preserved and $f$ is an automorphism of $G$.
	\end{proof}

\end{enumerate}

\end{document}
